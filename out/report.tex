%! Author = Deneth Weerasinghe
%! Date = 15/12/2021

% Preamble
\documentclass[11pt]{article}

% Packages
\usepackage{amsmath}
\usepackage{hyperref}
\usepackage{graphicx}
\usepackage{siunitx}

% Document
\begin{document}

    \begin{figure}
        \centering\includegraphics[scale=0.5]{LULogo}\label{fig:figure}
    \end{figure}

    \title{\textbf{PHYS281: Coding and Modelling Project Report}}
    \author{\textbf{Deneth Weerasinghe,}\\
    Lancaster University\\
    d.weerasinghe@lancaster.ac.uk}

    \date{
        \parbox{\linewidth}{
            \centering
            Github repository:\\
            \url{https://github.com/deneth-weerasinghe/PHYS281_Simulation-Project}\\
            \bigskip
            \today
        }}

    \maketitle

    \begin{abstract}
        \noindent
        In this report, we describe to which degree a simple, 3D n-body system of gravitating particles, written in Python, can simulate gravitation.
        This is done by modelling a simulation of the solar system as 10 point-like particles, consisting of the Sun, the planets and Pluto.
        The position of the particles are updated using a variety of integration methods, being Euler, Euler-Cromer, Euler-Richardson and Verlet, each with their sets of accuracies.
        Results of this simulation using each method is compared to real data of the positions and velocities of solar objects from NASA's Jet Propulsion Laboratory (JPL) dataset[1]x and further insight is gained from analysing the conserved quantities
        of total energy and linear momentum.

    \end{abstract}

    \tableofcontents


    \section{Introduction}\label{sec:intro}

    \subsection{Newtonian gravitation}\label{subsec:newtonian-gravitation}
    This project attempts to simulate the motion of any number of bodies using Newtonian gravity as its underlying model, thus operating under classical mechanics.\\
    In this model of gravitation, the gravitational force $\vec{F}$ acting on a body of mass $m_1$ due to a body of mass $m_2$ is proportional
    to the product of those masses and is a function of the distance $|\vec{r}|$ between those two bodies, as shown in the following\cite{newton1999principia}:
    \begin{equation}
        \vec{F} = \frac{Gm_{1}m_{2}}{|\vec{r}|}\hat{r}\label{eq:equation1}
    \end{equation}
    Here, $G$ is the constant of proportionality of this relation, known as the gravitational constant.
    This value has been derived empirically.
    One of the first times it was derived was via the 1798 Cavendish experiment, which indirectly determined the value of G as $6.74 \times 10^{-11} \si{m^{3}.kg^{-1}.s^{-1}}$\cite{cavendish}.\\
    As seen, this is an example of the inverse square law: the force does not scale linearly but rather decreases quadratically as distance between bodies increase.
    As such, the resulting gravitational field is radial and infinite;
    no matter how small the masses involved are or how large the distances between bodies are, the force of gravity still acts on an object.\\
    The gravitational acceleration is then simply derived from $\vec{F} = m\vec{a}$ as per Newton's second law.

    \subsection{Modelling and Limitations}\label{subsec:modelling}
    True for all simulations, there are abstractions to be used in order to greatly simplify the model.
    This will inevitably give rise to error when comparing the results to empirical data which will have to be taken into account when discussing results.\\
    The first abstraction is to model all gravitating objects as point-like particles with no physical dimensions.
    The simplification is justified in this model due to the scale used for the simulation.
    In celestial contexts such as the solar system, the lengths of object is negligible when compared to the vast distances between each of them.
    For instance, consider the radius of the largest body in this simulation, the Sun, 696,342 \si{km}\cite{2012ApJ...750..135E}.
    This is only 1.2\% of Mercury's semi-major axis of 57,909,050 \si{km}\cite{JPL2}.
    As a result, this leads to inaccurate results when objects enter the volume that would be occupied by another object, as collision is what should happen.\\
    Another abstraction is disregarding relativistic effects and continuing to use classical mechanics despite it being superseded by general relativity.
    This is justified by relativistic contributions being negligible as no object travels at relativistic speeds in this simulation.
    However, an observable incongruity with Newtonian mechanics is the precession of the perihelion of Mercury's orbit,
    which has been demonstrated to have been caused by the significant curvature of spacetime around the Sun\cite{1916AnP...354..769E}.
    This will not be reflected in the simulation.\\
    The simulation will also run using time steps (denoted as $\Delta t$).


    \subsection{Numerical Methods}\label{subsec:numerical-methods}
    In order to simulate the motion in a system, that is to calculate the positions, velocities and accelerations of all bodies at all times $t$,
    we need to estimate them using numerical methods for ordinary first-order differential equations.
    These are only approximations of the equations of motion and will give rise to inevitable margins of error, different for each of the methods used.
    \subsubsection{Euler}
    This is the Euler method

    \subsubsection{Euler-Cromer}
    This is the Euler-Cromer method

    \subsubsection{Euler-Richardson}
    This is the Euler-Richardson

    \subsubsection{Verlet}
    This is the Verlet Method

    \bibliographystyle{vancouver}
    \bibliography{report_bib}
\end{document}